% !TeX spellcheck = de_DE
% Die erste (unkommentierte) Zeile im Dokument legt immer die
% Dokumentklasse fest
\documentclass{scrreprt} 

% Präambel:
% Einbinen von zusätzlichen Paketen. Falls für eine Datei keine Endung
% explizit angegeben wird, benutzt LaTeX '.tex'. Im Folgenden wird
% also die Datei 'edv_pakete.tex' eingebunden.
% Die erste Zeile im Dokument legt immer die Dokumentklasse fest
%\documentclass[notitlepage]{scrreprt}
    % Die wichtigsten Dokumentklassen:
    %   scrbook, scrreprt, scrartcl, beamer, standalone
    % Einige gängige Optionen für \documentclass:
    %   ngerman
    %   titlepage, notitlepage
    %   onecolumn, twocolumn
    %   oneside, twoside
    %wird in Hauptdatei festgelegt

% Präambel

% Einige KOMA-Script-Optionen
\KOMAoptions{fontsize=12pt,paper=a4}      %Schriftgröße, Papierformat
\KOMAoptions{DIV=11}                      % Parameter mit dem man den Seitenrand ändern kann
\KOMAoptions{listof=totoc}

% Hier werden einige Pakete eingebunden
\usepackage[utf8]{inputenc}               % Direkte Eingabe von ä usw. Input=Eingabe
\usepackage[T1]{fontenc}                  % Font Kodierung für die Ausgabe Font=Ausgabe
\usepackage[ngerman]{babel}               % Verschiedenste sprach-spezifische Extras, ngerman für neue deutsche Rechtschreibung, auch UK oder US möglich
\usepackage[autostyle=true]{csquotes}     % Intelligente Anführungszeichen, arbeitet mit Babel zusammen
%

\usepackage{amsmath}%Mathedarstellung
\usepackage{commath}%Mathedarstellung
\usepackage{IEEEtrantools}%IEEEeqnarray
%
\usepackage{siunitx}   % Intelligentes Setzen von Zahlen und Einheiten
\sisetup{locale = DE}  % Deutsch als locale für die Zahlen und Einheiten
%http://tex.stackexchange.com/questions/2291/how-do-i-change-the-enumerate-list-format-to-use-letters-instead-of-the-defaul

\usepackage{enumitem}%erlaubt u.A. die Aufzählung mit Buchstaben, gefunden auf http://tex.stackexchange.com/questions/2291/how-do-i-change-the-enumerate-list-format-to-use-letters-instead-of-the-defaul
%
\usepackage[varg]{txfonts}                % Schönere Schriftart, muss nach amsmath, damit keine Fehlermeldung kommt
\usepackage{graphicx} %einbinden von Figuren/Bildern
%\graphicspath{{figs/}} % Stammverzeichnis der verwendeten Bilder, muss im selben Ordner wie Hauptdatei sein
%
\usepackage[backend=biber, style=numeric, sorting=none]{biblatex}
%Verwenden von \cite in \footnote: Bibliographie drucken lassen, mehrmals kompilieren
\usepackage{hyperref}%erzeugt klickbare Elemente
\usepackage[all]{hypcap}%hyperref-befehle springen zum oberen Rand des Bildes
% Zum Einbinden von Programmcode verwenden wir das listings-Paket
\usepackage{listings}

% Für Syntax-Highlighting:
\usepackage{xcolor}
%Fuer Bilder mit gnuplot

\usepackage{color}

% Die folgenden listings-Einstellungen sind nötig, um
% deutsche Umlaute und die Tilde (~) in listings-Umgebungen
% verwenden zu können.
\lstset{
    basicstyle=\ttfamily,    
    literate={~} {$\sim$}{1} % set tilde as a literal
    {ö}{{\"o}}1
    {ä}{{\"a}}1
    {ü}{{\"u}}1
    {ß}{{\ss}}1
    {Ö}{{\"O}}1
    {Ä}{{\"A}}1
    {Ü}{{\"U}}1
}

% Farben für Code-Syntaxhighlighting und Weiteres festlegen:
\lstset{
    % Keine besondere Markierung für Leerzeichen in Codes
    showspaces=false,               
    showstringspaces=false,         
    % Farebn für Code-Kommentare und Schlüsselworte:
    commentstyle=\color{red},       % comment style
    keywordstyle=\color{blue},      % keyword style
    stringstyle=\color{orange},		% string style
    breaklines=true,
    numbers=left,                    % where to put the line-numbers; possible values are (none, left, right)
    numbersep=5pt,                   % how far the line-numbers are from the code
    stepnumber=5, 					%how often there are line numbers in code listings
    tabsize=4, 						%default tabsize set to 4 spaces
    %language=python,
    }
%gefunden auf https://en.wikibooks.org/wiki/LaTeX/Source_Code_Listings
%eigene Kommandos/Abürzungen
\newcommand{\tb}{\textbackslash}
\newcommand{\txt}{\texttt}

%Soll Zeilenumbrueche in Gleichungen vermeiden
\binoppenalty=9999
\relpenalty=9999


% Verzeichnisse mit Abbildungen; kann gestrichen werden,
% falls Sie dies schon in edv_pakete.tex definiert haben:
\graphicspath{{Bilder}}
\addbibresource{refs.bib}

\usepackage{textcomp}
\usepackage{eurosym}

%\addbibresource{refs.bib} %Hinzufügen einer Literaturdatenbank aus dem angegebenen Verzeichnis

% Titel, Autor und Datum
\title{Monte-Carlo Simulation eines statistischen Modells auf einem Parallelrechner}
\date{Juli 2020}
\author{Christiane Groß}

% Jetzt startet das eigentliche Dokument
\begin{document}
	\maketitle
	
	% Römische Zahlen für die Seitennummern des Inhaltsverzeichnisses
	\pagenumbering{roman}
	
	% Inhaltsverzeichnis kommt hier
	\tableofcontents
	
	\clearpage
	
	% Normale Zahlen für die Seitennummern des Fliesstextes
	\pagenumbering{arabic}
	%Inhalt
	\chapter{Theoretischer Hintergrund}
	\section{Monte-Carlo Simulationen}
	\section{Das Ising-Modell}
	\chapter{Umsetzung}
	\section{Initialisierung und Thermalisierung}
	Gitter, Hamiltonian, Thermalisierungsschwelle
	
	Gitter mit zwei for-schleifen füllen, Mersenne-Twister, der auf +-1 zugeordnet wird.
	Hamiltonian durch Gitterdurchgang mit zwei Schleifen nach Formel $H=-J\sum_{<i,j>}S_iS_j$
	
	Sweep: geht gesamtes Gitter durch, dreht Spin um mit Wahrscheinlichkeit $min\left(1, exp(-\Delta H/T)\right) $
	
	Thermalisierung: 
	Vergleichen von hamiltonian vor und nach sweep, solange sweeps, bis Energie sich nicht mehr verringert. Bilder?
	
	Ausgabe mit x, y Position, Spin in txt datei, mit gnuplot als Heatmap plotten
	\section{Messungen}
	\subsection{Akzeptanzrate}
	Beim Sweep wird jede Veränderung gezählt, am Ende wird Veränderungen/laenge/laenge in datei geschrieben. Bild, erst gering, dann steil steigend, flacht ab, bildet Plateau
	Bei welchen Temperaturen?
	\subsection{Magnetisierung}
	Magnetisierung= Abs(Summe über alle Spins im Gitter)
	
	Bilder
	
	-Bei kleinen Temperaturen, unterhalb von T ungefähr 0,7, ist die
	 Magnetisierung konstant 1, mit sehr geringen Abweichungen.
	 
	 -Zwischen ca. 0,7 und ca. 2,5 nimmt die Magnetisierung ab, springt jedoch
	 immer wieder zwischen stärkeren und schwächeren Magnetisierungen hin und
	 her. Dies wird erst ab einer gewissen Gitterlänge sichtbar, bei kleineren
	 Gitterlängen (ausprobiert habe ich es mit 10), ist die Abnahme viel weniger
	 sprunghaft. In diesem Bereich ist die Standardabweichung der Messungen recht
	 groß.
	 
	 -Bei größeren Temperaturen, oberhalb von ca. 2,5, ist die Magnetisierung
	 annähernd konstant und fast null. Der relative Fehler ist in diesem Bereich
	 recht groß, jedoch ist der absolute Fehler, genau wie der Messwert, klein.
	\chapter{Ergebnisse}
	%Anhang
	\listoffigures
	\listoftables
	
	
	
	\printbibliography[heading=bibintoc]
\end{document}